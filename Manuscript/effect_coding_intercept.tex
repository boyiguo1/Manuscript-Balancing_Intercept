% Options for packages loaded elsewhere
\PassOptionsToPackage{unicode}{hyperref}
\PassOptionsToPackage{hyphens}{url}
%
\documentclass[
]{article}
\usepackage{amsmath,amssymb}
\usepackage{lmodern}
\usepackage{iftex}
\ifPDFTeX
  \usepackage[T1]{fontenc}
  \usepackage[utf8]{inputenc}
  \usepackage{textcomp} % provide euro and other symbols
\else % if luatex or xetex
  \usepackage{unicode-math}
  \defaultfontfeatures{Scale=MatchLowercase}
  \defaultfontfeatures[\rmfamily]{Ligatures=TeX,Scale=1}
\fi
% Use upquote if available, for straight quotes in verbatim environments
\IfFileExists{upquote.sty}{\usepackage{upquote}}{}
\IfFileExists{microtype.sty}{% use microtype if available
  \usepackage[]{microtype}
  \UseMicrotypeSet[protrusion]{basicmath} % disable protrusion for tt fonts
}{}
\makeatletter
\@ifundefined{KOMAClassName}{% if non-KOMA class
  \IfFileExists{parskip.sty}{%
    \usepackage{parskip}
  }{% else
    \setlength{\parindent}{0pt}
    \setlength{\parskip}{6pt plus 2pt minus 1pt}}
}{% if KOMA class
  \KOMAoptions{parskip=half}}
\makeatother
\usepackage{xcolor}
\IfFileExists{xurl.sty}{\usepackage{xurl}}{} % add URL line breaks if available
\IfFileExists{bookmark.sty}{\usepackage{bookmark}}{\usepackage{hyperref}}
\hypersetup{
  pdftitle={Lean on your statistics: The generalization and simplification of the balance intercept problem},
  pdfauthor={Boyi Guo, Jacqueline Rudolph},
  hidelinks,
  pdfcreator={LaTeX via pandoc}}
\urlstyle{same} % disable monospaced font for URLs
\usepackage[margin=1in]{geometry}
\usepackage{graphicx}
\makeatletter
\def\maxwidth{\ifdim\Gin@nat@width>\linewidth\linewidth\else\Gin@nat@width\fi}
\def\maxheight{\ifdim\Gin@nat@height>\textheight\textheight\else\Gin@nat@height\fi}
\makeatother
% Scale images if necessary, so that they will not overflow the page
% margins by default, and it is still possible to overwrite the defaults
% using explicit options in \includegraphics[width, height, ...]{}
\setkeys{Gin}{width=\maxwidth,height=\maxheight,keepaspectratio}
% Set default figure placement to htbp
\makeatletter
\def\fps@figure{htbp}
\makeatother
\setlength{\emergencystretch}{3em} % prevent overfull lines
\providecommand{\tightlist}{%
  \setlength{\itemsep}{0pt}\setlength{\parskip}{0pt}}
\setcounter{secnumdepth}{-\maxdimen} % remove section numbering
\newlength{\cslhangindent}
\setlength{\cslhangindent}{1.5em}
\newlength{\csllabelwidth}
\setlength{\csllabelwidth}{3em}
\newlength{\cslentryspacingunit} % times entry-spacing
\setlength{\cslentryspacingunit}{\parskip}
\newenvironment{CSLReferences}[2] % #1 hanging-ident, #2 entry spacing
 {% don't indent paragraphs
  \setlength{\parindent}{0pt}
  % turn on hanging indent if param 1 is 1
  \ifodd #1
  \let\oldpar\par
  \def\par{\hangindent=\cslhangindent\oldpar}
  \fi
  % set entry spacing
  \setlength{\parskip}{#2\cslentryspacingunit}
 }%
 {}
\usepackage{calc}
\newcommand{\CSLBlock}[1]{#1\hfill\break}
\newcommand{\CSLLeftMargin}[1]{\parbox[t]{\csllabelwidth}{#1}}
\newcommand{\CSLRightInline}[1]{\parbox[t]{\linewidth - \csllabelwidth}{#1}\break}
\newcommand{\CSLIndent}[1]{\hspace{\cslhangindent}#1}
\usepackage{array}
\ifLuaTeX
  \usepackage{selnolig}  % disable illegal ligatures
\fi

\title{Lean on your statistics: The generalization and simplification of
the balance intercept problem}
\author{Boyi Guo, Jacqueline Rudolph}
\date{}

\begin{document}
\maketitle

As the field of epidemiology evolves, there are growing interests to
employ more computational approaches to solve analytic problems. Among
them, simulation is one of the most accessible concepts. Previous
literature argues the importance of simulation in epidemiology education
and research. {[}TODO: add citations{]} Even though computational tools
can be very helpful, we caution the excess reliance on the computation
in analytic problem solving and the total neglect of fundamental
statistics theories. In the article, we demonstrate how basic statistics
knowledge can simplify the balance intercept problem and provide a
generalized solution. Specifically, we dissect the balance intercept
problem from a statistical perspective, formulate the balance intercept
calculation for all outcomes and link functions, and provide an
alternative solution that require significantly less computation.

The balance intercept concept was first introduced by Rudolph et al.
(2021) to control the marginal probability of binary outcomes when
constructing a simulation study. The authors proposed to calculate a
``balance intercept'' to replace the ``standard intercept'' in
simulation procedures. Robertson, Steingrimsson, and Dahabreh (2021)
discovered that the analytic solution of the balance intercept produces
inaccurate control when the data generating model employs a logistic
link function. They proposed to use root finding algorithms to find the
balance intercept numerically. Later, Zivich and Ross (2022) generalized
the numeric solutions to several other outcome distributions.

To better understand the balance intercept problem, we first review two
basic statistics concepts, coding schemes and link functions. A coding
scheme describes how a categorical variable is enumerated in a
regression system. Most commonly used coding schemes include the
reference coding (also known as dummy coding) and the effect coding.
Both coding schemes create \(p-1\) columns for a categorical variable
with \(p\) levels. The reference coding employs 0 and 1 to denote the
level an individual belongs, while the effect coding employs 0, 1, and
-1. For example, a binary variable would be coded as a vector of 0 for
the reference level and 1 for the comparing level using the reference
coding and a vector of 1 and -1 using the effect coding. (See an example
of three-level categorical variable in Table 1). As the enumerations are
different for the two coding schemes, they offer different
interpretations of regression coefficients. The reference coding
emphasizes the change relative to a reference level of preference; the
effect coding emphasizes the deviation from the grand mean (here refer
to as the mean of the level means). Nevertheless, the two schemes
translate to each other one-on-one and provide the same statistical
inference. To note, both coding schemes can be applied ubiquitously in
any regression system regardless the outcome distribution and the link
function of choice. In addition, a link function describes the
mathematical relationship between linear predictor and the mean of the
outcome. Similar to generalized linear models, the choice of estimands
decides the link function in the data generating model. For example, if
the estimands of observation is risk difference, an identify function is
preferred instead of a logistic function in the data generating model.
The choice of the link function in the data generating model has impact
on the accuracy of the imbalance intercept approximation and will be
elaborate more later.

\begin{table}
\label{tab:code_scheme}
\centering
\begin{tabular}{ccc}
\hline
Levels & Referece & Effect\\
\hline
Level 1 & (0, 0) &  (1, 0)\\
        & $\beta_0$ & $\beta_0 + \beta_1$\\
Level 2 & (1, 0) &  (0, 1)\\
        & $\beta_0 + \beta_1$ & $\beta_0 + \beta_2$\\
Level 3 & (0, 1) &  (-1, -1)\\
        & $\beta_0 + \beta_2$ & $\beta_0 - \beta_1 - \beta_2$ \\
\hline
\end{tabular}
\caption{One-on-one translation and coefficient interpretation of the reference and effect coding for a three-level categorical variable.}
\end{table}


The original proposal (Rudolph et al. 2021) of the balance intercept
defaults to the reference coding system without explicit mentioning, and
carries along in the subsequent studies. The intercept term in the
reference coding scheme describes the conditional probability of the
reference level of a categorical variable. Forcing the intercept term to
the target marginal probability (referred to as the standard intercept
in Rudolph et al. (2021)) would naturally fail due to the definition
discrepancy. Statistically, to derive the balance intercept is to derive
the conditional probability of the reference level and can be achieved
with analytic and numeric approximations. Nevertheless, the accuracy
could be vulnerable to the quality of numeric analysis and programming
proficiency.

The primary reason that the analytic approximation (Rudolph et al. 2021)
of the balance intercept may fail is because of the inequality between a
function of the expectation and the expectation of a function when the
function is not linear. Derive the analytic approximation on the linear
predictor scale as in Rudolph et al.~(2021) will introduce this problem
when link functions is nonlinear. (Robertson, Steingrimsson, and
Dahabreh 2021) Hence, the derivation of the balance intercept should be
conducted on the response scale, e.g., the probability scale for binary
outcomes regardless of the link function. Following the mathematical
derivation in the Supporting Material {[}TODO: add hyperlink{]}, it is
clear Rudolph's balance intercept equation is the exact solution for
linear link functions. We also provide a closed-form solution for
logarithm link function. Another commonly used link function, the logit
function, is not easily tractable, and hence is recommended to use Monte
Carlo algorithms or root-find algorithms to derive numerically.

Using the reference coding system in data generating models is very
convenient when the comparison between the group means are the known
information. Nevertheless, it requires additional calculation when
conditional probabilities are given. In this case, it is more convenient
to use effect coding system, where the coefficients describe the
discrepancy of the level means. Meanwhile, the intercept term in the
effect coding scheme describes the mean of the group means, which
coincides with the marginal probability when the sample sizes are
balanced across levels. In other words, the balance intercept problem
does not exist for simulations with balanced designs when using the
effect coding. When the underlying design is not balanced, the intercept
is simply to calculate the grand mean by averaging the conditional
probabilities, \(\delta_0 = \sum\limits_{j=1}^p \pi_j /p\).

In this report, we provide a statistics perspective to the balance
intercept problem. Specifically, we clarify the simulation procedure
when the reference coding scheme and the effect coding scheme are used
as the foundation of the procedure. With an statistical analysis of the
simulation procedure, we provide a generalized solution to calculate the
balance intercept that addresses all forms of outcomes and link
functions. We show that the balance intercept problem drastically
simplifies when applying effect coding scheme. Notably, when the design
of the study is balanced, there is minimum calculation is needed.

In addition to the statistical solution for the balance intercept
problem, we want to emphasize that the fundamental skills of statistics
can not be ignored, even in the erra of computation. The growing
computation power can greatly reduce the technical burden to derive
analytic solutions with numeric devices. Nevertheless, the accuracy of
numeric solution greatly depends on the perfection of the
implementation, and can be easily overlooked. The fundamental statistics
skills can provide a shortcut to the correct solution, and provide great
translatability across programming languages in comparison to numeric
solutions. The author doesn't arugula if analytic approach and
computational approach is superior. Instead, we advocate for a balanced
emphasis on both computational skills and analytic thinking.

\hypertarget{references}{%
\section*{References}\label{references}}
\addcontentsline{toc}{section}{References}

\hypertarget{refs}{}
\begin{CSLReferences}{1}{0}
\leavevmode\vadjust pre{\hypertarget{ref-robertson2021}{}}%
Robertson, Sarah E, Jon A Steingrimsson, and Issa J Dahabreh. 2021.
{``Using Numerical Methods to Design Simulations: Revisiting the
Balancing Intercept.''} \emph{American Journal of Epidemiology},
November, kwab264. \url{https://doi.org/10.1093/aje/kwab264}.

\leavevmode\vadjust pre{\hypertarget{ref-rudolph2021}{}}%
Rudolph, Jacqueline E, Jessie K Edwards, Ashley I Naimi, and Daniel J
Westreich. 2021. {``SIMULATION IN PRACTICE: THE BALANCING INTERCEPT.''}
\emph{American Journal of Epidemiology} 190 (8): 1696--98.
\url{https://doi.org/10.1093/aje/kwab039}.

\leavevmode\vadjust pre{\hypertarget{ref-zivich2022}{}}%
Zivich, Paul N, and Rachael K Ross. 2022. {``RE: {``}Using Numerical
Methods to Design Simulations: Revisiting the Balancing
Intercept{''}.''} \emph{American Journal of Epidemiology}, May.
\url{https://doi.org/10.1093/aje/kwac083}.

\end{CSLReferences}

\end{document}
